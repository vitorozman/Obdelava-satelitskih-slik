\documentclass[12pt]{article}
\usepackage[utf8]{inputenc}
\usepackage[slovene]{babel}

\usepackage{hyperref}
\usepackage{listings} % za naslovnico
\usepackage{amsthm}
\usepackage{amsmath, amssymb, amsfonts}
\usepackage{graphicx}
%\graphicspath{ {./Slike/} }
\usepackage{subcaption} % za side-by-side slike
\usepackage[
top    = 2.cm,
bottom = 2.cm,
left   = 2.cm,
right  = 2.cm]{geometry}

\usepackage{footnote}
\makesavenoteenv{tabular}
\title{Seminarska naloga ITAP - 1.del}

\begin{document}
    
\author{Vito Rozman}
\date{\today}
\maketitle

%\subsection*{Podatki}
\textbf{Podatki} 
Prvi korak je bil zmašanje število vrstic, ker pri velikem številu podatkov računalnik ni 
prenesel računske zahtevnosti. Iz vseh podatkov z referenco sem vzel $1\%$ podatkov, nato sem jih razdelil na 
učno in testno množico v razmerju $3:1$, torej $75\%$ učna in $25\%$ testna.

\textbf{Metrika}
Za primerjavo modelov sem izbral metriko natančnosti prečnega 
preverjanja, natančnost na testni množici in ploščina pod ROC krivuljo (AUC). Pri modelu za 
napovedovanje pozidanega območja sem primerjal tudi glede občutljivosti in specifičnosti.   

\textbf{Modeli}
Primerjal sem modele \emph{glm}, \emph{knn}, \emph{svm} in \emph{rf}. Med primerjavo sem optimiziral še parametre pri 
\emph{knn} (število sosedov) 
in pri \emph{svm} (kompleksnost). Dobil sem naslednje rezultate:

\begin{center}
    \begin{tabular}{||c| c c c c||} 
     \hline
     Model - gozd & GLM & KNN & SVM & RF\\ [0.5ex] 
    \hline\hline
    Natančnost - testna	& 0.9739 & 0.9717 &	0.9772 & 0.9770\\

     \hline
     Natačnost - cv	&0.9737 & 0.9715 & 0.9756 & 0.9764\\
     \hline
     Ploščina pod ROC (AUC)	& 0.9955	& 0.9903 & 0.9957 & 0.9963 \\ 
     \hline
    \end{tabular}
\end{center}
\begin{center}
    \begin{tabular}{||c| c c c c||} 
        \hline
        Model - pozidano & GLM & KNN & SVM & RF\\ [0.5ex] 
       \hline\hline
       Natančnost - testna	& 0.9242&	0.9016&	0.9362&	0.9377\\
   
        \hline
        Natačnost - cv	&0.9195	&0.9204 & 0.9271&	0.9355\\
        \hline
        Ploščina pod ROC (AUC)	& 0.9651 & 0.9235 &	0.9702 & 0.9793\\ 
        \hline
        Občutljivost	& 0.8863 & 0.9722 &	0.9218 & 0.9262\\ 
        \hline
        Specifičnost	& 0.9327 & 0.9213 &	0.9353 & 0.9483\\ 
        \hline
       \end{tabular}
\end{center}

\textbf{Najboljši model:}
Kot vidno zgoraj se je v obeh primerih najbolje izkazal \emph{rf} (naključni gozdovi), saj je pri najpomembnejših metrikah dobil najboljše 
rezultate. Po izbiri modela sem optimiziral parametra mtry in ntree. Edina slaba lastnost \emph{rf} se je pokazala 
pri časovni zahtevnosti in težavi pri velikem številu vrstic. V primeru da bi optimiziral še časovno
komponento, bi izbral \emph{glm} ali pa \emph{svm}.

Za napovedovanje pozidanega območja, sem moral podatke malo prilagoditi, ker bi se lahko pojavila težava
pri prekomernem napovedovanju negativnih izidov, saj je razmerje pozidanega (pozitivni) proti nepozidanemu 
(negativni) precej manjša kot pri podatkih za gozdove. Uporabil sem metodo  podvzorčenja.

Končni model za napoved gozd sem natreniral na $20\%$ vseh podatkov z referenco (ker mi računalnik ni 
dopuščal več), za napoved pozidanega območje pa sem vzel $90\%$ vseh podatkov z referenco na katerih sem uporabil 
podvzorčenje.


\end{document}